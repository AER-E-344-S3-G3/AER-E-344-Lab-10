\chapter{Discussion}
\label{cp:discussion}

Questions need to be answered:

\begin{itemize}
  \item What can you say about the predicted pressures in comparison to the measured pressures? That is, does the theory under or over‐predict the wall pressure?
  Overall, the graphs of the measured pressure values matched the shape and form of the theoretical values (see \autoref{fig:measured_vs_theoretical_pressure}, but the theoretical pressures under-predict the measured static and total pressures for the 1st Critical Condition, 2nd Critical Condition, and when there is a normal shock in the nozzle. However, the measured total pressure was much lower than the predicted theoretical value for the 3rd Critical Condition.
  \item Give some possible reasons for the differences.
  The theory does not consider factors that lead to a difference between the measured pressure and the predicted pressure. Assumptions made by the theory include ideal gas behavior, steady-state conditions, and a frictionless flow. Also, the theory ignores boundary layer effects, which greatly influence the flow behavior near the nozzle's walls. Lastly, the flow can present non-uniformity, affecting the pressure measurements. As the Mach number of the flow increases, these factors have a greater impact on the flow characteristics, leading measured values to be more distant from theoretical values.
  \item What might this mean for the prediction of other flow quantities such as Mach number, temperature, etc.?
  The derivation of the other flow quantities, such as the Mach number and the temperature, uses the values measured for pressure. Thus, these quantities will also tend to have different values from those calculated from the theory.
  \item Point out any interesting anomalies you might see in the measured data.
  3rd Critical Condition total pressure
\end{itemize}


  