\chapter{Conclusion}
\label{cp:conclusion}
To analyze the effects of a converging-diverging nozzle in supersonic conditions, we measured the wall static pressure of the nozzle and used the Schlieren Technique to identify data at the 1st, 2nd, and 3rd Critical Conditions as well as when a normal shock exists in the nozzle. Knowing the flow is choked at those conditions, the static pressure data was used to find the mach number and total pressure across the nozzle. Using the known cross-sectional area at any point on the nozzle and its ratio to the throat area, the theoretical mach number and pressure distribution were found and compared with the measured data.