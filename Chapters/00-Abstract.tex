\thispagestyle{plain} % Page style without header and footer
% \pdfbookmark[1]{Resumo}{resumo} % Add entry to PDF
% \chapter*{Resumo} % Chapter* to appear without numeration
% \blindtext

% \keywordspt{Keyword A, Keyword B, Keyword C.}

% \blankpage

\pdfbookmark[1]{Abstract}{abstract} % Add entry to PDF
\chapter*{Abstract} % Chapter* to appear without numeration

Efficient propulsion design relies on understanding supersonic flow within nozzles, often predicted by simplified \acrfull{1d} theories that estimate performance factors like form drag through assumed flow dynamics. However, the reliability of these predictions rallies on the accuracy of the assumptions made. This experiment tests these assumptions by measuring wall pressures in a de Laval nozzle at various flow states, namely: under-expanded, third critical condition, over-expanded, second critical condition, normal shock in the nozzle, and first critical condition—identified using Schlieren imaging. By correlating these real-world measurements with theoretical equations, this experiment bridges the gap between theory and practice, enhancing the predictive reliability of these simplified flow theories. 

% \blankpage


