\chapter{Introduction}
\label{cp:introduction}
Understanding how supersonic flow behaves inside rocket nozzles is crucial for designing efficient propulsion systems. Engineers often use simplified theories, like one-dimensional (1D) nozzle flow theory, to predict things like form drag – the pressure difference inside and outside the nozzle that affects performance. However, these theories rely on a lot of assumptions about how the flow works. So, it is important to check if these assumptions are reliable by testing them in real-world experiments.

In this experiment, we focused on measuring wall pressures along a de Laval nozzle under different conditions. We looked at six scenarios that consisted of Under‐expanded, 3rd critical, Over‐expanded, 2nd critical, Normal shock existing inside the nozzle, and 1st critical flows. Each of the operating conditions had its own unique shock characteristics. To figure out what conditions we are dealing with, we used images from Schlieren images to spot the shock patterns. Then, we will pressure along the nozzle using the data acquisition software. 




